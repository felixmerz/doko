\documentclass[11pt,a4paper,landscape]{article}
\usepackage[utf8]{inputenc}
\usepackage[a4paper,margin=0.5in]{geometry}

\usepackage{auto-pst-pdf}
\usepackage{pst-poker}
\usepackage{multicol}
\psset{inline=symbol}
\pagestyle{empty}
\setlength\parindent{0pt} % noindent
\begin{document}

\begin{center}
\textbf{Doppelkopf - Soloübersicht}
\end{center}

\begin{multicols}{3}
%% karo = d, herz = h, pik = s, kreuz = c

%Text

\textbf{Allgemeines} Der Solist spielt alleine gegen die drei anderen Spieler und wählt zu Beginn das entsprechende Solo. Bei einem \textit{Pflichtsolo} spielt der Solist auf. Bei einem \textit{Lustsolo} spielt - wie gewohnt - der Spieler links vom Geber auf. \\
Der Solist bildet alleine die \textit{Re}-Partei. An-/Absagen können nach der üblichen Regel getätigt werden. Die Sonderpunkte für \textit{Fuchs fangen}, \textit{Doppelkopf} und \textit{Charlie} entfallen. Am Ende der Auswertung werden die Punkte des Solisten verdreifacht (egal ob bei Sieg oder Niederlage). 

\begin{center}
\textbf{Folgende Soli können gespielt werden} \\
\end{center}

\textbf{Farbsolo} Im Vergleich zum Normalspiel ersetzen die Karten \textit{Ass}, \textit{Zehn} (außer beim Herzsolo), \textit{König} und \textit{Neun} in der gewählten Farbe das \textit{Karo-Ass}, die \textit{Karo-Zehn}, den \textit{Karo-König} und die \textit{Karo-Neun} als \textit{Trumpf}. Die ersetzten \textit{Karo} werden hierbei zu einer neuen \textit{Fehl} in üblicher Rangfolge. Beim \textit{Karosolo} wird also nichts geändert und entspricht der Rangfolge wie im Normalspiel (\textit{Trumpfsolo}). \textit{Dullenregel} und \textit{Schwein} gelten beim Farbsolo weiterhin und werden auch nicht an die Farbwahl angepasst.

Als Beispiel ein \textit{Kreuzsolo}: \\
\hspace*{3.65cm}\textbf{Trumpf} \\
\vspace*{-0.3cm} \\
\hspace*{0.38cm}
\begin{pspicture}(-0.5,-3)(3,3)
\psset{crdshadow=none}
\uput{0}[0](0,0){\crdninec}
\uput{1}[0](0,0){\crdKc}
\uput{2}[0](0,0){\crdtenc}
\uput{3}[0](0,0){\crdAc}
\uput{4}[0](0,0){\crdJd}
\uput{5}[0](0,0){\crdJh}
\uput{6}[0](0,0){\crdJs}
\uput{7}[0](0,0){\crdJc}
\uput{8}[0](0,0){\crdQd}
\uput{9}[0](0,0){\crdQh}
\uput{10}[0](0,0){\crdQs}
\uput{11}[0](0,0){\crdQc}
\uput{12}[0](0,0){\crdtenh}
\end{pspicture} \\
\hspace*{3.9cm}\textbf{Fehl} \\
\vspace*{-0.3cm} \\
\begin{pspicture}(1,-3)(3,3)
\psset{crdshadow=none}
\uput{0}[0](0,0){\crdnined}
\uput{1}[0](0,0){\crdKd}
\uput{2}[0](0,0){\crdtend}
\uput{3}[0](0,0){\crdAd}
\uput{6.2}[0](0,0){\crdnineh}
\uput{7.2}[0](0,0){\crdKh}
\uput{8.2}[0](0,0){\crdAh}
\uput{11.4}[0](0,0){\crdnines}
\uput{12.4}[0](0,0){\crdKs}
\uput{13.4}[0](0,0){\crdtens}
\uput{14.4}[0](0,0){\crdAs}
\end{pspicture} 
\vfill\null
\columnbreak

Für die folgenden \textbf{weiteren Soli} werden im Allgemeinen die vier \textit{Fehl} (Farben) in folgender aufsteigender Rangfolge genutzt: \\ 

\hspace*{1.8cm}\textbf{Karo}\hspace{3.3cm}\textbf{Herz} \\
\vspace*{-0.3cm} \\
\hspace*{0.18cm}
\begin{pspicture}(0.3,-3)(3,3)
\psset{crdshadow=none}
\uput{0}[0](0,0){\crdnined}
\uput{1}[0](0,0){\crdJd}
\uput{2}[0](0,0){\crdQd}
\uput{3}[0](0,0){\crdKd}
\uput{4}[0](0,0){\crdtend}
\uput{5}[0](0,0){\crdAd}
\uput{8.7}[0](0,0){\crdnineh}
\uput{9.7}[0](0,0){\crdJh}
\uput{10.7}[0](0,0){\crdQh}
\uput{11.7}[0](0,0){\crdKh}
\uput{12.7}[0](0,0){\crdtenh}
\uput{13.7}[0](0,0){\crdAh}
\end{pspicture}

\hspace*{1.85cm}\textbf{Pik}\hspace{3.4cm}\textbf{Kreuz} \\
\vspace*{-0.3cm} \\
\hspace*{0.18cm}
\begin{pspicture}(0.3,-3)(3,3)
\psset{crdshadow=none}
\uput{0}[0](0,0){\crdnines}
\uput{1}[0](0,0){\crdJs}
\uput{2}[0](0,0){\crdQs}
\uput{3}[0](0,0){\crdKs}
\uput{4}[0](0,0){\crdtens}
\uput{5}[0](0,0){\crdAs}
\uput{8.7}[0](0,0){\crdninec}
\uput{9.7}[0](0,0){\crdJc}
\uput{10.7}[0](0,0){\crdQc}
\uput{11.7}[0](0,0){\crdKc}
\uput{12.7}[0](0,0){\crdtenc}
\uput{13.7}[0](0,0){\crdAc}
\end{pspicture} \\
\vspace*{0.2cm}\\
Je nach gewähltem Solo, wird aus den entsprechenden Karten der \textit{Trumpf} (in aufsteigender Reihenfolge) gebildet. \textit{Dullenregel} und \textit{Schwein} entfallen.

Jedes dieser \textit{Soli} kann dabei auch in umgedrehter Rangfolge gespielt werden (\textit{umgedreht}). \\ \textit{Trumpf} schlägt dabei immer noch \textit{Fehl}. \\

\textbf{Fleischlos} Kein Trumpf - es kann daher lediglich \textit{Fehl} bedient oder abgeworfen werden.  \\

\textbf{Reines Farbsolo} Eine der \textit{Fehl} wird als Trumpf gewählt. \\

\hspace*{0.45cm}\textbf{Bubensolo}\hspace{1.05cm}\textbf{Damensolo}\hspace{0.95cm}\textbf{Königssolo} \\
\vspace*{-0.3cm} \\
\begin{pspicture}(1.3,-3)(3,3)
\psset{crdshadow=none}
\uput{0}[0](0,0){\crdJd}
\uput{1}[0](0,0){\crdJh}
\uput{2}[0](0,0){\crdJs}
\uput{3}[0](0,0){\crdJc}
\uput{6.05}[0](0,0){\crdQd}
\uput{7.05}[0](0,0){\crdQh}
\uput{8.05}[0](0,0){\crdQs}
\uput{9.05}[0](0,0){\crdQc}
\uput{12.1}[0](0,0){\crdKd}
\uput{13.1}[0](0,0){\crdKh}
\uput{14.1}[0](0,0){\crdKs}
\uput{15.1}[0](0,0){\crdKc}
\end{pspicture} 
\vfill\null
\columnbreak
\hspace*{3.55cm}\textbf{Hurenhaus} \\
\vspace*{-0.3cm} \\
\hspace*{1.98cm}
\begin{pspicture}(-2.5,-3)(3,3)
\psset{crdshadow=none}
\uput{1}[0](0,0){\crdJd}
\uput{2}[0](0,0){\crdJh}
\uput{3}[0](0,0){\crdJs}
\uput{4}[0](0,0){\crdJc}
\uput{5}[0](0,0){\crdQd}
\uput{6}[0](0,0){\crdQh}
\uput{7}[0](0,0){\crdQs}
\uput{8}[0](0,0){\crdQc}
\end{pspicture} 

\hspace*{3.75cm}\textbf{Edelpuff} \\
\vspace*{-0.3cm} \\
\hspace*{1.98cm}
\begin{pspicture}(-2.5,-3)(3,3)
\psset{crdshadow=none}
\uput{1}[0](0,0){\crdQd}
\uput{2}[0](0,0){\crdQh}
\uput{3}[0](0,0){\crdQs}
\uput{4}[0](0,0){\crdQc}
\uput{5}[0](0,0){\crdKd}
\uput{6}[0](0,0){\crdKh}
\uput{7}[0](0,0){\crdKs}
\uput{8}[0](0,0){\crdKc}
\end{pspicture} 

\hspace*{3.2cm}\textbf{Mönchskloster} \\
\vspace*{-0.3cm} \\
\hspace*{1.98cm}
\begin{pspicture}(-2.5,-3)(3,3)
\psset{crdshadow=none}
\uput{1}[0](0,0){\crdJd}
\uput{2}[0](0,0){\crdJh}
\uput{3}[0](0,0){\crdJs}
\uput{4}[0](0,0){\crdJc}
\uput{5}[0](0,0){\crdKd}
\uput{6}[0](0,0){\crdKh}
\uput{7}[0](0,0){\crdKs}
\uput{8}[0](0,0){\crdKc}
\end{pspicture} 

\hspace*{3.45cm}\textbf{Bilderbuch} \\
\vspace*{-0.3cm} \\
\hspace*{0.98cm}
\begin{pspicture}(-0.5,-3)(3,3)
\psset{crdshadow=none}
\uput{1}[0](0,0){\crdJd}
\uput{2}[0](0,0){\crdJh}
\uput{3}[0](0,0){\crdJs}
\uput{4}[0](0,0){\crdJc}
\uput{5}[0](0,0){\crdQd}
\uput{6}[0](0,0){\crdQh}
\uput{7}[0](0,0){\crdQs}
\uput{8}[0](0,0){\crdQc}
\uput{9}[0](0,0){\crdKd}
\uput{10}[0](0,0){\crdKh}
\uput{11}[0](0,0){\crdKs}
\uput{12}[0](0,0){\crdKc}
\end{pspicture} \\

\textbf{Sokrates} Es werden zwei der drei einfachen Bildsoli (\textit{Buben-/Damen-/Königssolo}) in einer Reihenfolge gewählt und zu Beginn angesagt. Die erste Hälfte der Stiche gilt das erste \textit{Solo}, in der zweiten Hälfte wird zum zweiten \textit{Solo} gewechselt. \\



\end{multicols}
\end{document}
